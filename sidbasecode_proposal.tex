\documentclass[10pt]{article}
\usepackage[utf8]{inputenc}
\usepackage{url}
\usepackage{hyperref}
\usepackage{amsmath}
\usepackage{amsfonts}
\usepackage{amssymb}
\usepackage{graphicx}
\usepackage{float}
\usepackage{lipsum}
\usepackage{multicol}
\usepackage{xcolor}
\usepackage{natbib}
\usepackage[font=small]{caption}
\addtolength{\abovecaptionskip}{-3mm}
\addtolength{\textfloatsep}{-5mm}
\setlength\columnsep{20pt}

\usepackage[a4paper,left=1.50cm, right=1.50cm, top=1.50cm, bottom=1.50cm]{geometry}

\author{Josh Bailey <josh@vandervecken.com>}

\title{Sequencing the Chiptune Genome}

\begin{document}

        \begin{center}
          {\Large \textbf{Sequencing the Chiptune Genome}}\\
                \vspace{1em}
                Proposal for Thesis\\
                \vspace{1em}
                \textit{
                Ph.D. Student (Music) - Josh Bailey,
                Student Number: 300665818\\
                  New Zealand School of Music—Te Kōkī,
                  Victoria University of Wellington}\\
                \vspace{1em}
                Supervisors\\
                \vspace{1em}
                \textit{
                  Jim Murphy, Victoria University of Wellington\\
                  Dugal McKinnon, Victoria University of Wellington}\\
        \end{center}

        \begin{center}
                \rule{150mm}{0.2mm}
        \end{center}

        \begin{abstract}
          An early 8 bit home computer, the Commodore 64, contained
          capable sound synthesis hardware for the the period, but no
          built in software tools to program it with.  Composers were
          required to write low level machine code to fully access the
          sound hardware's features. This resulted in musical
          performances such as game sound tracks, being realized as
          machine code programs, including the ``patches'' (synthesis
          pipeline configuration from moment to moment) and
          score. Over time composers invented new patches and borrowed
          or improved upon the work of others by reverse engineering
          machine code and experimenting with hardware programming
          techniques not anticipated by the manufacturer.  This
          process continues today in the form of a competitive
          ``demoscene'' community.

          It is proposed to make a musicological enumeration and study
          of all known Commodore 64 sound programming techniques,
          detecting and attributing the invention of each technique
          and its variations over time. This would be done by
          automating the analysis of a large collection of Commodore
          64 music in the High Voltage SID Collection (over 55,000
          performances), with software decompilation and reverse
          engineering tools developed specifically for the purpose.

          Beyond the musicological results, machine learning tools
          would be developed to detect the machine language
          programming techniques characteristic of given composers and
          to generate new machine language programs. These tools may
          inform and advance current discussions on the role of
          generative music and attribution of human composers
          contributions to techniques in AI training datasets. While
          the Commodore 64 is a limited platform, its limitations
          facilitate a complete investigation of its capabilities, and
          may inform further research on automated composer attribution
          generally.
        \end{abstract}

        \begin{center}
                \rule{150mm}{0.2mm}
        \end{center}

        \vspace{5mm}

\begin{multicols*}{2}

\section{Introduction}

%% \subsection{Formatting}

%% The formatting instructions are directed to applicants for the 2024--2025 Bloomberg Data Science Ph.D. Fellowship Application. All applicants are required to adhere to the formatting specifications described in this document. These specifications were created with the intent of providing an equitable format and ensure that applications are similar in length and format.

%% Applicants should adhere to all the formatting instructions described in this document. \textbf{Failure to do so will result in the Fellowship application not being reviewed.} In particular, the should be no changes to page size, spacing, margins and font size as to what is specified in this document.
%% Proposals must be in two-column format, with the exception of the title, applicant name and abstract, which must be centered at the top of the first page, and any full-width figures or tables.

%% Applicants are required to provide a Portable Document Format (PDF) version of their proposal in the SoftConf application website (\url{https://www.softconf.com/n/Bloomberg2024/}), together with the CV and the referee names. \textbf{The Fellowship proposal should be no more than two pages of content for PhD Fellowship applications, plus unlimited pages for references.}

%% All figures and tables that are part of the proposal should fit the page limit. A sample figure is presented in Figure~\ref{fig:fig1}. A sample table that includes how to cite references is presented in Table~\ref{tbl:tbl1}. We do not allow for submission of additional material such as appendices and supplementary materials like data or code.

%% \subsection{Structure}

%% The following sections described in this template can be used as a guide to structure the proposal, but will not be enforced.

%% You can use the \texttt{Introduction} section to:

%% \begin{itemize}
%%     \item briefly describe the broad research area of the proposal;
%%     \item the limitations of current research this proposal aims to address;
%%     \item the motivation and impact of the work planned in this research proposal;
%%     \item any additional background required for understanding the proposal.
%% \end{itemize}

\begin{table*}
        \centering
        \begin{tabular}{cc}
                \hline
                \textbf{Citation format} & \textbf{Citation command} \\
                \hline
                \citet{APA:83} & \textbackslash{}citet{} \\
                \citep{APA:83} & \textbackslash{}citep{} \\
                \hline
        \end{tabular}
        \caption{This is sample table with full page width.}
        \label{tbl:tbl1}
\end{table*}

\begin{figure}[H]
    \centering
        \includegraphics[width=\columnwidth]{example-image}
        \caption{This is a sample figure.}
        \label{fig:fig1}
\end{figure}

\section{Methodology}

\section{Preliminary work}

\section{Thesis structure, timeline, plan for dissemination}

\section{Novel contributions and conclusions}

\end{multicols*}

\clearpage

\bibliography{nzsm-phd-proposal}
\bibliographystyle{plainnat}

\end{document}
