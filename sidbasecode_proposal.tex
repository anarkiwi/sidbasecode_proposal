\documentclass[10pt]{article}
\usepackage[utf8]{inputenc}
\usepackage{url}
\usepackage{hyperref}
\usepackage{amsmath}
\usepackage{amsfonts}
\usepackage{amssymb}
\usepackage{graphicx}
\usepackage{float}
\usepackage{lipsum}
\usepackage{multicol}
\usepackage{xcolor}
\usepackage{natbib}
\usepackage[font=small]{caption}
\addtolength{\abovecaptionskip}{-3mm}
\addtolength{\textfloatsep}{-5mm}
\setlength\columnsep{20pt}

\usepackage[a4paper,left=1.50cm, right=1.50cm, top=1.50cm, bottom=1.50cm]{geometry}


\author{}

\title{2023--2024 Bloomberg Data Science Ph.D. Fellowship Application}

\begin{document}
	
	\begin{center}
		{\Large \textbf{Proposal Title for 2024--2025 Bloomberg Data Science Ph.D. Fellowship}}\\
		\vspace{1em}
		{\large Applicant name}\\
		\vspace{1em}
		\textit{Affiliation}
	\end{center}
	

	\begin{center}
		\rule{150mm}{0.2mm}
	\end{center}		

	\begin{abstract}
	
	This document contains the instructions for preparing a valid 2024--2025 Bloomberg Data Science Ph.D. Fellowship Application research proposal. This document conforms to its own specifications and is an example of what your proposal should look like.
	
	The abstract should briefly describe the proposal and make the case for the proposal to non-experts in your particular field. The rest of the proposal should be directed at a technical data science audience.
	
	\textbf{Collaborators}: list the names and affiliations of expected collaborators on the project here
	\end{abstract}

	\begin{center}
		\rule{150mm}{0.2mm}
	\end{center}		

	\vspace{5mm}
	
\begin{multicols*}{2}

\section{Introduction}
 
\subsection{Formatting} 
 
The formatting instructions are directed to applicants for the 2024--2025 Bloomberg Data Science Ph.D. Fellowship Application. All applicants are required to adhere to the formatting specifications described in this document. These specifications were created with the intent of providing an equitable format and ensure that applications are similar in length and format.

Applicants should adhere to all the formatting instructions described in this document. \textbf{Failure to do so will result in the Fellowship application not being reviewed.} In particular, the should be no changes to page size, spacing, margins and font size as to what is specified in this document.
Proposals must be in two-column format, with the exception of the title, applicant name and abstract, which must be centered at the top of the first page, and any full-width figures or tables.

Applicants are required to provide a Portable Document Format (PDF) version of their proposal in the SoftConf application website (\url{https://www.softconf.com/n/Bloomberg2024/}), together with the CV and the referee names. \textbf{The Fellowship proposal should be no more than two pages of content for PhD Fellowship applications, plus unlimited pages for references.}

All figures and tables that are part of the proposal should fit the page limit. A sample figure is presented in Figure~\ref{fig:fig1}. A sample table that includes how to cite references is presented in Table~\ref{tbl:tbl1}. We do not allow for submission of additional material such as appendices and supplementary materials like data or code.

\subsection{Structure} 

The following sections described in this template can be used as a guide to structure the proposal, but will not be enforced.

You can use the \texttt{Introduction} section to:

\begin{itemize}
    \item briefly describe the broad research area of the proposal;
    \item the limitations of current research this proposal aims to address;
    \item the motivation and impact of the work planned in this research proposal;
    \item any additional background required for understanding the proposal.
\end{itemize}

\begin{table*}
	\centering
	\begin{tabular}{cc}
		\hline
		\textbf{Citation format} & \textbf{Citation command} \\
		\hline
		\citet{APA:83} & \textbackslash{}citet{} \\
		\citep{APA:83} & \textbackslash{}citep{} \\
		\hline
	\end{tabular}
	\caption{This is sample table with full page width.}
	\label{tbl:tbl1}
\end{table*}

	
\begin{figure}[H]
    \centering
	\includegraphics[width=\columnwidth]{example-image}
	\caption{This is a sample figure.}
	\label{fig:fig1}
\end{figure}

\section{Proposed Work}

This section can be used to describe the technical steps needed to achieve the stated aim of the research proposal.
	
\section{Expected Results \& Impact} 
	
This section can be used to describe the expected outcomes of the research proposal and their broader impact to the research field and beyond.
	
\section{Data, Software and Ethics Policy}

Briefly state the means by which the results produced by this funding will be disseminated e.g., open source software, open access journals, conferences, and/or presentations.

Please mention if there are any licensing or release constraints for the data used.

If applicable, please discuss or address any ethical considerations of the proposal including: the use of the data, ethical approvals needed for using the data, how access to data used will be provided and intended uses of the resulting methods.

Finally, references are unlimited in length.


\paragraph{Bloomberg employee consultants:} Please disclose if you consulted with any Bloomberg employee while preparing this application.

\end{multicols*}

\clearpage

\bibliography{bb-ds-fellowship}
\bibliographystyle{plainnat}
	
\end{document}
